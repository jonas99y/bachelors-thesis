\section{Phase 2}
Der in der ersten Phase erstellte Prototyp wurde dem Auftraggeber in Form eine Demo gezeigt.
Dabei sind folgenden Rückmeldungen entstanden:
\begin{itemize}
    \item Der Prototyp entspricht den Vorstellungen und erfüllt die definierten Anforderungen.
    \item Die Vorgabe, dass die Benutzerschnittstelle für Mobilgeräte ausgelegt sein soll, fällt weg.
          Sie soll nur für Desktop-Geräte ausgelegt sein.
    \item Nebst Spannungswerten soll die Applikation auch Leistungs- und THD Werte verarbeiten können.
    \item Bisher wurden sämtliche Daten künstlich erstellt.
    Diese sollen um echte Messwerte erweiter werden, welche von Auftraggeber zur Verfügung gestellt werden.
    \item Die Benutzerschnittstelle soll um die Funktion erweitert werden,
    dass eine spezifische Zeitspanne ausgewählt werden kann.

    

\end{itemize}

\subsection{Fakemeter}

\subsection{\ac{MQTT} Datenverarbeitung}

\subsection{Benutzerschnittstelle}

\subsection{Schwierigkeiten}
Schwierigkeiten während der zweiten Phase
