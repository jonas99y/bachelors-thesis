\section{Sprint 2}
Spillover von sprint1
-> Erster release erstellen Hauptziel
Lesen soll für alles funktionieren.
Einfach Typen funktionieren, für array und structure muss noch eine Lösung erarbeitet werden.

CI/CD
Build funktioniert, jedoch noch keine tests

Deploy:
Das Deployen funktioniert, jedoch crasht de anwendung beim User direkt.
vermutlich brechtigungsproblem

\subsection{Build bei Commit}\label{s2:buildOnCommit}
spillover
<RuntimeIdentifiers>win10-x86;win10-x64;win10-arm64</RuntimeIdentifiers> vs <RuntimeIdentifiers>win-x64</RuntimeIdentifiers>
Probleme beim Build via msbuild Kommando und im CI/CD

Probleme beim ausführen von tests via vstest.console.exe
"Could not find testhost"
Dieses Problem tritt ebenfalls auf, wenn ein neues Testprojekt erstllt wird und dies mit vstest abgespielt wird.

Lösung:
dotnet cli verwenden. neuses Problem:  The imported project "C:\\Program Files\\dotnet\\sdk\\6.0.200\\Microsoft\\DesktopBridge\\Microsoft.DesktopBridge.props" was not found.
Lösung:
neue Solution nur mit Test projekten

\subsection{Integration Tests}
https://devblogs.microsoft.com/ifdef-windows/winui-desktop-unit-tests/
-> nicht nötig, Integration tests ohne UI schreiben
Als direkt VM Objekte verwenden.


Sehr praktisch. UI must nicht mehr gestartet werden um Kommunikation mit Zähler zu testen.





\subsection{Erster Release}
Als alle Spillover Stories von Sprint 1 implementiert waren, wurde eine erste Version der Software veröffentlicht.
Dazu wurde für das Team eine kurze Demo gemacht und gezeigt was bereits funktioniert und was noch nicht. 
Die Resonanz war dabei positiv. Die optische Erscheinung der Benutzerschnistelle wurde gelobt.
Da die Funktionen dieser Versionen noch recht eingeschränkt sind, war nicht damit zu rechnen, dass sie von den Teammitgleideren direkt eingesetzt wird.



\subsection{Schreiben und Ausführen}
Die neuen Stories des zweiten Sprints beinhalten das Schreiben von Attributen sowie das ausführen von Methoden.
Aufgrund der geleisteten Vorarbeiten betreffend Lesen von Attributen konnte das Schreiben mit geringem Aufwand implementiert werden.
Das zuvor entwickelte Design konnte gut erweitert werden.

Es stellte sich heraus, dass beim Ausführen einer Methode lediglich ein Schreibbefehl ausgeführt wird.
So musste lediglich der Code für das Schreiben von Attributen etwas angepasst werden und in der Benutzerschnistelle entsprechend aufgerufen werden.



\subsection{InfraLib}
Parsen von ClassDescriptions kaum getestet. Xml node hat im Code einen anderen Namen als in den Dateien.

Neue tests für InfraLib schreiben und code fixen.

\subsection{Adapter Pattern}

Mithilfe des Adapter Pattern wurde die bestehende Klassenstrukture der InfraLib auf die benötigten Interfaces adaptiert.
Dies erlaubte es, die Interfaces genau so zu modellieren, wie sie für den DlmsQuickAccess benötigt werden.
Dependency auf InfraLib ist an einem Ort.


\subsection{Sprint Abschluss}
Der Sprint konnte bereits einen Arbeitstag vor dem geplanten Ende abgeschlossen werden.