\section{Sprint 3}
Der Fokus des dritten Sprints liegt bei der Verbesserung der Benutzererfahrung.
Das Ziel ist es, dass die Anwendung danach für die täglichen Arbeiten der Entwickler eingesetzt werden kann.

Die grösste Lücke besteht aktuell noch darin, dass es nicht möglich ist, zwischen verschiedenen Zähler-Modellen zu wechseln.
Es ist lediglich möglich mit dem Referenzprodukt "Ref\_MMI3" zu kommunizieren.

Des Weiteren sollen kleinere Verbesserungen an der Benutzerschnittelle gemacht werden.


\subsection{Produkte}
Benötigte Infos:
- Object Model xml file
- HW file
   - hat link auf ObjectList file

Object Model aus Artifacts des Nightly
+ Möglichkeit für eigene
HW file aus Picasso\_Test

vs.

alles mit der Anwendung bundeln?
- cons:
weniger Flexiblität
abhänig vom Netz
abhänig von offiziellen Object Models, keine manuelle anpassungen für Debugging möglich
Unnötig viele Updates nötig
- pros:
einfach umzusetzten
auto update kümmert sich darum, dass immer alles aktuell ist.

: Kann ja zusätzlich immernoch gemacht werden?


ObjectModel kann aus Repo geladen werden, ObjectList ebenfalls.
evtl. sogar ats file?
Wo ist source of truth von bswp: DLMS\_OPTICAL\_HW1\_Ref\_MMI3.xml?


Dlms HW file nicht standardisiert abgelegt
-> Diese fix an den release binden mit Möglichkeit für Erweiterung

.dlmsquickaccess file in repo that directly opens the app?
-pros:
File will be created via workspaces or manually.
This file could then also be used to store preferences of the user. Its up to the user, where and how he stores it.

HW could either be specified by a path or by on of the predefined HW files.
app would save previous opened configs.



\subsection{Schwierigkeiten}
FilePicker nicht eifach zu verwenden
https://github.com/microsoft/WindowsAppSDK/issues/1188

Kommunikation muss deinitialisiert werden (TLS). Destruktor wird nicht aufgerufen von GC. 
In diesem Sprint ein Hack angewendet, da im nächsten asynchrone kommunikation ansteht.