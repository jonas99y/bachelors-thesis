\section{Usability}
Ins Deutsche übersetzt bedeutet Usability nach \citeauthor{usability} (\citeyear{usability}) Gebrauchstauglichkeit oder Nutzerfreundlichkeit.
Sie sorgt dafür, dass Programme einfach verwendet werden können.
Gemessen wird sie daran, wie schnell und direkt ein Benutzer sein Ziel während der Nutzung der Software erreicht.
Die Norm DIN EN ISO 9241 beschreibt einige Eigenschaften, welche eine Anwendung mit hoher Usability haben soll.
Diese sind:
\begin{itemize}
   \item Der Aufgabe angemessen
   \item Selbst beschreibend
   \item Steuerbar
   \item Erwartungskonform
   \item Fehlertolerant
   \item Lernförderlich
   \item Motivierend
\end{itemize}
Um diese Eigenschaften zu erreichen, ist ein entsprechendes Vorgehen währen der Entwicklung nötig.
Dies wir im nächsten Abschnitt erläutert.


\subsection{Projektablauf mit Usability als Ziel}
\citeauthor{usability} (\citeyear{usability}) sagen, dass die Nutzer während der Entwicklung möglichst früh eingebunden werden.
So kann von ihnen die Bestätigung eingeholt werden, dass das entwickelte gefällt.
Wenn es nicht gefällt, so kann wieder von vorne angefangen werden.
Laut \citeauthor{usability} ist in einem optimalen Projektablauf der Nutzer in jeder Phase eingebunden.
So soll in einer ersten Phase der Nutzer kennengelernt und verstanden werden.
Dazu werden im folgenden Abschnitt einige Methoden erklärt.
In einer nächsten Phase sollen die Anforderungen der Nutzer spezifiziert werden.
Diese sollen in der dritten Phase anhand von Prototypen umgesetzt werden und in der letzten Phase von den Nutzern evaluiert werden.
Agile Vorgehensmodelle eignen sich gut, um regelmässig Tests mit den Nutzer durchzuführen, beispielsweise nach jedem Sprint.


\subsection{Methoden um die Nutzer kennen zu lernen}
Um Anforderungen an die Anwendung formulieren zu können, sollen zuerst die Nutzer dieser Anwendung kennengelernt und deren Bendürfnisse sowie Wünsche erfasst werden.
\citeauthor{usability} (\citeyear{usability}) nennen dazu mehrere Methoden, diese werden hier vorgestellt:

\subsubsection{Fokusgruppen}\label{fokusgruppe}
Bei einer Fokusgruppe handelt es sich um eine Gruppendiskussion.
Ein Moderator führt mit fünf bis zehn Personen ein Gespräche zu einem bestimmten Thema.
So können bestehende Anwendungen verbessert oder Ideen für neue gesammelt werden.
Ein erstelltes Konzept kann den Nutzern erstmals gezeigt und von ihnen evaluiert werden.
Der Austausch innerhalb der Fokusgruppe soll gefördert werden, da dies die Kreativität anregt.
Es kann hilfreich sein, wenn sich die Teilnehmer der Diskussion im Voraus auf diese vorbereiten.
So sind sie gezwungen, bereits im Voraus eine Meinung zu bilden und es kann vermieden werden, dass Teilnehmer der Gruppenmeinung anschliessen.


\subsubsection{Befragungen}\label{befragung}
Ziel einer Befragung ist es, die individuellen Meinungen von Einzelpersonen zu erfassen.
In der Regel werden sie mithilfe eines Fragebogens durchgeführt.
Im Gegensatz zu den Fokusgruppen sind sie eher dazu da, um quantitative Daten zu erheben und weniger für die Erfassung neuer Ideen geeignet.
Wir die selbe Befragung über längere Zeit mehrmals durchgeführt, so können Veränderungen oder Trends gemessen werden.


\subsubsection{Vor-Ort-Beobachtungen}\label{vorort}
Bei einer Vor-Ort-Beobachtung wird der Benutzer aus dem Hintergrund bei einer typischen Nutzungssituation beobachtet.
Er soll dabei vom Beobachter möglichst wenig beeinflusst werden, so dass seine Abläufe möglichst alltäglich sind.
Der Beobachter notiert wie die Software verwendet wird.
Dabei soll speziell auf unbewusstes Verhalten der Nutzer geachtet werden.


\subsubsection{Tagebuchstudien}\label{tagebuchstudien}
Bei Tagebuchstudien erfassen die Nutzer ihr eigenes Verhalten.
Entweder bei der Nutzung einer konkreten Anwendung oder bei Situationen bei denen eine geplante Anwendung eingesetzt werden soll.
So können Erlebnisse der Nutzer über einen längeren Zeitraum gesammelt werden.
Ereignisse und Erfahrungen welche unregelmässig oder gar einmalig auftreten können so erfasst werden.
Dies ist bei Vor-Ort-Beobachtungen nur schwer möglich.