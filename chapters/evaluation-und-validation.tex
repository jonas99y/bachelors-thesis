% Auswertung und Interpretation der Ergebnisse. Nachweis, dass die Ziele erreicht wurden, oder
% warum welche nicht erreicht wurden.

\chapter{Evaluation und Validation}\label{eval}
In diesem Kapitel wird ausgewertet, ob die Ziele der Arbeit erreicht wurden.
Um die Funktionalität des DlmsQuickAccess zu beurteilen werden Rückmeldungen der Nutzer aufgelistet und analysiert.
Für einen Vergleich mit den \ac{DMT2} werden diese ebenfalls beigezogen.
Des Weiteren wird betrachtet, wie die Schwerpunkte Softwarequalität und Usability umgesetzt wurden.



\section{Rückmeldungen der Nutzer}\label{feedback}
Im Verlauf des sechsten Sprints (\ref{s6}) wurden von den Nutzer der Anwendung ausführliche Rückmeldungen abgeholt.
Der nächste Abschnitt widmet sich den positiven Rückmeldungen.
Danach werden jene aufgelistet, welche von gefundenen Fehlern und möglichen Verbesserungen handeln.
\subsection{Positive Rückmeldungen}

\begin{enumerate}
   \item Die Funktion, einzelne Objekte oder Attribute als Favorit vermerken zu können, ist sehr hilfreich.
   \item Da die Beschreibungen der Enum Werte direkt angezeigt werden, bleibt das separate öffnen der Class Description erspart.
   \item Es ist praktisch, dass beim Starten der Anwendung direkt das gewünschte Produkt angewählt werden kann. Beim Starten des \ac{DMT2} ist es nicht klar, welche Konfiguration gerade geladen ist.
   \item Die Anwendung wirkt auf mich optisch ansprechend und modern.
   \item Die Lese- und Schreibgeschwindigkeit ist sehr eindrücklich. Ein klarer Vorteil gegenüber \ac{DMT2}.
   \item Ich nutze den \ac{DMT2} nur noch dann, wenn ein bestimmtes Attribut im DlmsQuickAccess nicht richtig funktioniert. Ich hoffe, dass die Anwendung von der Landis+Gyr den nötigen Support erhält und weiterentwickelt wird.
\end{enumerate}

\subsection{Gefundene Fehler und Verbesserungsvorschläge}\label{gefundeneFehler}
Rückmeldungen, welche bereits umgesetzten werden konnten wurden in dieser Auflistung weggelassen.
\begin{enumerate}
   \item Beim Auslesen von grossen Arrays wirkt die Benutzerschnittstelle unübersichtlich.
   \item Es wäre praktisch, wenn ausgelesene Werte aus der Anwendung exportiert werden könnten, z.B. als \ac{YAML} Datei.
   \item Attribute vom Type \textit{CHOICE} funktionieren nicht richtig.
   \item Die Titel von Objekten werden je nach Länge nicht richtig dargestellt.
   \item Beim Schreiben von Zahlenwerten muss der Nutzer die Länge des Typs kennen und beachten, dies ist mühsam und fehleranfällig.
   \item Gelesene Arrays werden teilweise falsch dargestellt.
   \item Es wäre hilfreich, wenn die Suche mit einer Tastenkombination gestartet werden könnte.
\end{enumerate}

\section{Vergleich mit DMT2}
Im Abschnitt \ref{dmt} wurden die Stärken und Schwächen des \ac{DMT2} aufgezählt.
Diese werden nun für einen Vergleich mit dem DlmsQuickAccess genutzt.

Die Stärke, dass der \ac{DMT2} einfach konfiguriert werden kann, wird im DlmsQuickAccess weitergeführt.
Das Design, dass der DlmsQuickAccess mit einer bestimmten Produktkonfiguration gestartet werden muss, ist bei den Nutzern beliebt.
Dadurch ist es auch möglich, mehrere Instanzen mit unterschiedlichen Konfigurationen gleichzeitig zu verwenden.
Dies ist beim \ac{DMT2} nicht möglich.
Beim \ac{DMT2} ist es möglich, die aktuelle Konfiguration mittels Benutzerschnittstelle zu bearbeiten.
Dies ist beispielsweise dann nützlich, wenn der COM-Port des angeschlossen Zählers geändert werden muss.
Beim DlmsQuickAccess ist es nicht möglich, die Konfiguration zu ändern, nachdem die Anwendung gestartet wurde.
Dies ist jedoch auch kaum nötig, da z.B. der COM-Port jeweils automatisch richtig gesetzt wird.

Im DlmsQuickAccess werden die \ac{COSEM}-Objekte mit Informationen aus den Class Descriptions ergänzt.
Dies ist beim \ac{DMT2} nicht der Fall.
Die Nutzer müssen diese Verknüpfung jeweils mit Hilfe der Word Dokumente manuell und im Kopf durchführen.

Im \ac{DMT2} werden Anfragen und Antworten jeweils mit einer XML Repräsentation dargestellt.
Diese ist beim Lesen umständlich und beim Schreiben fehleranfällig.
Im DlmsQuickAccess wurde auf eine solche Darstellung verzichtet.
Keiner der Nutzer meldete zurück, dass er diese vermisst, obwohl bei der Umfrage zu Beginn des Projekts (Anhang \ref{anhang:survey}) noch über die Hälfte der Nutzer diese als nützlich bezeichneten.

Ein Schwäche des \ac{DMT2} ist, dass nur die Antwort der zuletzt durchgeführten Kommunikation angezeigt wird.
Dies konnte im DlmsQuickAccess verbessert werden, indem die ausgelesenen Werte bei den jeweiligen Attribut angezeigt werden.
Das heisst, ein gelesener Wert beleibt solange erhalten, bis das zugehörige Attribut erneut ausgelesen wird.
Eine Funktion, um alle durchgeführten Kommunikationen aufzuzeichnen und darzustellen wurde von den Nutzern sehr hoch priorisiert, konnte aus Zeitgründen jedoch nicht innerhalb der sechst Sprints umgesetzt werden.
Im Abschnitt \ref{ausblick} ist dies festgehalten.



\section{Softwarequalität}\label{evalQuality}
In Abschnitt \ref{kriterien} wurden acht Kriterien beschrieben, welche die Qualität einer Software messen.
Die Erfüllung dieser Kriterien wird in den folgenden Abschnitten beschrieben.

\subsection{Funktionalität}
Im Abschnitt \ref{feedback} sind Rückmeldungen der Nutzer aufgeführt.
Diese besagen, dass einige Teile der Anwendung bereits gut funktionieren und andere noch fehlerhaft sind.
Die Funktionalität lässt sich somit noch verbessern.

\subsection{Performance}
Wie im Abschnitt \ref{qualityPerformance} beschrieben, sind die Anforderungen an die Performance von Desktop Anwendung meist einfach zu erfüllen.
Dies ist auch beim DlmsQuickAccess der Fall.
Die Anwendung läuft flüssig und Befehle werden jeweils sofort ausgeführt.
Einzig das Scrollen durch die Liste von Objekten ist nicht immer ganz flüssig.

\subsection{Zuverlässigkeit}
SonarQube bewertet die Reliability der Software mit der Bestnote \dq A\dq.
In Abschnitt \ref{gefundeneFehler} sind Fehler aufgeführt, welche von den Nutzer gefunden wurden.
Um die Zuverlässigkeit der Anwendung noch weiter zu verbessern, müssen diese behoben werden.

\subsection{Benutzbarkeit}
Die Benutzbarkeit der Anwendung wird im Abschnitt \ref{evalUsabilty} beurteilt.

\subsection{Wartbarkeit}
SonarQube bewertet die Maintainability der Software mit der Bestnote \dq A\dq.
Dabei ist jedoch wichtig zu beachten, dass der \ac{ATS} Code nicht mitbewertet wurde.
Die Abhängigkeit zu diesem verschlechtert die Wartbarkeit.
Wenn ein Fehler in Kommunikationscode auftritt, kann es schwierig sein, diesen zu finden und zu beheben.

\subsection{Transparenz}
Anhand der Cognitive Complexity Metrik kann beurteilt werde, wie transparent der Code ist.
SonarQube meldet 13 Code Smells, welche von zu hoher kognitiver Komplexität kommen.
Diese befinden sich jedoch alle in Codeteilen, welche sich zwar im Repository befinden, jedoch im DlmsQuickAccess nicht verwendet werden (z.B. \textit{InfraLib}).
Somit ist die Transparenz der Software gut!

\subsection{Übertragbarkeit}
Der Code des \ac{ATS}, sowie jener der Benutzerschnittstelle sind an Windows gebunden.
Diese Teile der Anwendung lassen sich nicht auf andere Systeme übertragen.
Sie sind jedoch so vom Rest der Anwendung isoliert, dass sie einfach ersetzt werden könnten.
Die Übertragbarkeit der Anwendung ist somit weder sehr gut noch sehr schlecht.

\subsection{Testbarkeit}
Die zyklische Komplexität gibt an, wenn wie Komplex einzelne Methoden sind.
SonarQube meldet keine einzige Überschreitung des Grenzwertes bei dieser Metrik.
Da die Methoden der Anwendung nicht zu komplex sind, lassen sie sich auch besser testen.
Des Weiteren führte das Vorgehen nach \ac{TDD} dazu, dass Klassendesigns, welche nur schwierig zu testen sind, schnell erkannt wurden und verbessert werden konnten.


\section{Usabilty}\label{evalUsabilty}
Eigenschaften, welche eine Anwendung mit hoher Usability haben soll wurden in Abschnitt \ref{usability} aufgezählt.
Anhand dieser wird die Usability der Anwendung evaluiert.



TODO
Verschieden Kategorien?

short cuts

Fehlertoleranz anfangs noch sehr schlecht.

Erwartungskonform, da Object Model den Nutzer vertraut ist.

Alle information direkt ersichtlich

Farben sind schlicht.

Fehlermeldungen noch verbessert werden.
Im Log vorhanden, dem Nutzer noch nicht gut dargestellt.

Unterschiedliche Fenstergrössen unterstützen,