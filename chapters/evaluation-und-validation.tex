% Auswertung und Interpretation der Ergebnisse. Nachweis, dass die Ziele erreicht wurden, oder
% warum welche nicht erreicht wurden.

\chapter{Evaluation und Validation}

vergleich DMT2

implemntieret Features
Kundenfeedback

SonarQube

UI nach iso...