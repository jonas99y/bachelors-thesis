\chapter{Evaluation und Validation}
Um zu validieren, dass die Implementierungen den Anforderungen und Vorstellungen des Auftraggebers entsprechen,
wurden Entwicklungsfortschritte in mehreren Demonstrationen gezeigt.
Die Rückmeldungen der ersten beiden Demonstrationen werden
in den Kapiteln \ref{i2} Iteration 2 und \ref{i3} Iteration 3 jeweils am Anfang ausführlich beschrieben.

Das Feedback war meist positiv, führte jedoch auch dazu, dass die Benutzeroberfläche nach der ersten Iteration
komplett neu gestaltet wurde, da der Fokus von Mobile auf Desktop geschoben wurden.


In der Demo nach der dritten Iteration konnte dem Auftraggeber die finale Version der Applikation gezeigt werden.
Die dabei entstandenen Rückmeldungen beziehen sich somit nicht nur auf diese eine Iteration, sondern auch auf das gesamte Projekt.
Es wurde auch explizit abgeholt, ob die Anforderungen erfüllt wurden und die gelieferten Produkte den Erwartungen entsprechen.
Folgende Punkte wurden dabei vom Auftraggeber gesagt:
\begin{itemize}
    \item Die Funktionen für das Labeling der Daten sind verständlich.
    \item Alle Use Cases der Aufgabenstellung sind vollständig umgesetzt.
    \item Er möchte die Applikation gerne selbst testen. Dazu wäre eine Anleitung hilfreich.
    \item Wie die Applikation bei der Landis+Gyr AG eingesetzt werden soll, wird intern noch abgeklärt.
\end{itemize}


\section{Reflexion}
In den folgenden Abschnitten reflektieren die beiden Mitglieder des Projektteams das vergangene Projekt persönlich.
\subsection{Jonas Wyss}
Dieses Projekt war für mich eine gute Mischung aus Anwenden von Gelerntem und bereits Bekanntem
sowie auch Erlernen von Neuem.
Zu Beginn des Projekts war ich mit der Herausforderung konfrontiert, mit React.js die Grundsteine der
Benutzerschnittstelle zu implementieren. Dies war meine erste richtige Anwendung von React.js.
Anfangs musste ich dazu noch viel nachlesen und lernen.
Stundenlanges Suchen nach kleinen Fehlern stand auf der Tagesordnung und führte oft zu Frustration und Verzweiflung.
Ich bin froh, dass sich dies im weiteren Verlauf des Projekts gebessert hat und ich React.js mittlerweile als
eine erfreuliche Erweiterung meines persönlichen Programmier-Werkzeugkastens bezeichnen kann,
auf welche ich in Zukunft gerne zurückgreifen werde.

Für mich war es interessant, die Thematik der Strommessung aus einer neuen Perspektive zu bearbeiten.
In meinem Beruf programmiere ich für die Landis+Gyr AG an der Firmware der Stromzählern.

In der Zusammenarbeit mit Benjamin konnte ich von seiner Expertise in den Bereichen Python, Containern und CI/CD profitieren.
Wir haben uns gut ergänzt und konnten einander fordern.
Mit der gelieferten Applikation konnten wir die Aufgabenstellung erfüllen 
und unsere Fähigkeiten, welche wir während des Studiums erlernt haben, demonstrieren.

\subsection{Benjamin Fassbind}

Grundsätzlich bin ich sehr zufrieden mit dem erreichten Ergebnis.
Persönlich finde ich eine gute Architektur zu entwickeln, die sowohl simpel
als auch im nötigen Masse erweiterbar ist, eine grosse Herausforderung.
Aus meiner Erfahrung wird eine Architektur oft komplizierter als nötig.
Genau aus diesem Grund bin ich sehr zufrieden mit der entstandenen Architektur.
Sie ist nur so kompliziert wie nötig und die Komplexität dient dem Zweck
der Automatisierung was bei der Entwicklung und dem Deployment wieder einfacher wird.

Zudem war es für mich das erste Mal, dass ich ein richtiges Multicontainerprojekt
mit podman aufgesetzt habe. Mit den in diesem Projekt gemachten Erfahrungen
würde ich jederzeit wieder podman anstatt docker-compose für mein Setup wählen.

Auch die Python Library poetry, sqlmodel und FastAPI habe ich im Kontext dieses
Projektes zum ersten Mal benutzt. Zwar hab ich schon viel Erfahrung mit
ähnlichen Libraries, es ist aber eine willkommene Abwechslung bei einem
Greenfield Projekt neue Technologien auszuprobieren und zu sehen welche Probleme sie lösen
und ob sie wirklich besser sind als ihre Vorgänger.

Nicht zuletzt war es für mich ein spannender Einblick in die Welt von Landis+Gyr
und Stromzähler. Ich durfte viele Sachen in dieser Domäne im Verlaufe des Projektes lernen.

Gerne hätte ich noch etwas Zeit ins Frontend investiert, dies war aber aus zeitlichen Gründen
nicht mehr möglich.