\chapter{Methoden}

% Aus Aufbau_Bericht.pdf
% Hier halten Sie fest und begründen, welches Vorgehensmodell Sie für Ihr Projekt wählen. Sie
% verweisen allenfalls auf die daraus entstandenen, konkreten Terminpläne mit Meilensteinen, welche
% z.B. unter Realisierung (Kapitel 5) oder im Anhang versorgt sind.

% Bei Engineering-Projekten halten Sie weitere einzusetzende fachliche Methoden oder Techniken fest.
% Bei einem Softwareprojekt können dies z.B. der geplante Einsatz einer Anforderungsanalyse, der
% Einsatz von Review-Techniken (Architektur-Reviews) oder bekannter Programmiertechniken sein.



\section{Vorgehensmodell}\label{vorgehen}
In der Aufgabenstellung TODO cite wurde verlangt, das der Scrum Prozess, so wie er bei der Landis+Gyr gelebt wird, als Vorgehensmodell verwendet werden soll.
Diverse Eigenschaften und Abläufe dieses Prozesses konnten direkt für diese Arbeit übernommen und angewandt werden.
Da an dieser Arbeit nur ein Entwickler arbeitet und auch die Anzahl Stakeholders kleiner ist, wurde mehrere Dinge angepasst.
Die folgenden Abschnitte geben einen Überblick über den Prozess, wo sich dieser von jeden der Landis+Gyr unterscheidet und wieso diese Anpassungen vorgenommen wurden.

\subsection{Azure DevOps}\label{methoden:ADO}

\subsection{Workitems}
So ist es bei LG:
Epic / Feature / Story / Task


\subsection{Sprints}
Ebenfalls zwei wochen. Andere start/end tage.
Kapazität des Sprints stark von der Verfügbarkeit des einen Entwicklers abhänig.


\subsection{Definition of Done}
review nicht möglich


\subsection{Coding Standard}