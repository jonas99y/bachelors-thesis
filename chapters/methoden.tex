\chapter{Methoden}
Für dieses Projekt wurde zu Beginn ein agiles Vorgehen gewählt.
Der Grund dafür war, dass das genaue Projektziel noch nicht klar definiert werden konnte.
Als erstes sollte eine minimaler Prototyp erarbeitet werden. In der Aufgabenstellung \cite{aufgabenstellung} wurden dafür folgende Aufforderungen definiert: 
\begin{itemize}
\item Visualisierung des Spannungsverlaufs (drei Phasen) eines Stromzählers
basierend auf gespeicherten Messdaten.
\item Empfangen (TLS Verschlüsselung) und abspeichern von Messdaten (Spannung über drei Phasen) und
Zeitstempel mittels MQTT.
\item Bereitstellen von virtuellen Stromzählern, welche mittels MQTT Messdaten
übermitteln.
\end{itemize}
Nach ca. einem Drittel der Projektzeit wurde das erarbeitete dem Auftraggeber vorgeführt und die implementierten Funktionalitäten demonstriert.
Gemeinsam mit dem Auftraggeber wurden danach die Anforderungen für die nächste Iteration definiert.
Diese Prinzip wurde noch ein weiteres mal durchgeführt. 
Gegen Ende der Projektzeit wurde das Endresultat vom Projektteam präsentiert und eine letzte Rückmeldung von Auftraggeber eingeholt. 

% Aus Aufbau_Bericht.pdf
% Hier halten Sie fest und begründen, welches Vorgehensmodell Sie für Ihr Projekt wählen. Sie
% verweisen allenfalls auf die daraus entstandenen, konkreten Terminpläne mit Meilensteinen, welche
% z.B. unter Realisierung (Kapitel 5) oder im Anhang versorgt sind.

% Bei Engineering-Projekten halten Sie weitere einzusetzende fachliche Methoden oder Techniken fest.
% Bei einem Softwareprojekt können dies z.B. der geplante Einsatz einer Anforderungsanalyse, der
% Einsatz von Review-Techniken (Architektur-Reviews) oder bekannter Programmiertechniken sein.


\section{Teststrategie}
UI tests manuell. evtl. noch ein Testplan schrieben?

backend automatisiert mit e2e tests
% Dazu gehört auch eine Teststrategie (wo setzen Sie im Projekt Schwerpunkte betr. Testen?). Die
% eigentliche Testdurchführung ist dann unter Realisierung, im Anhang oder einem selbstständigen
% Testdokument beschrieben.