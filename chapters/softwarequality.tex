% https://dvmarcilio.github.io/papers/icpc2019.pdf
% http://aagasc.edu.in/cs/books/Software%20Quality%20Assurance%20From%20Theory%20to%20Implementation.pdf
% see document in Downloads!

\section{Softwarequalität}\label{softwarequality}
In der DIN-ISO-Norm 9126 wird Software Qualität so definiert:
\dq Software-Qualität ist die Gesamtheit der Merkmale und Merkmalswerte eines Software-Produkts, die sich auf dessen Eignung beziehen, festgelegte Erfordernisse zu erfüllen.\dq

\citeauthor{hoffmann2013software} (\citeyear{hoffmann2013software}) hebt zu dieser Definition hevor, dass es nicht ein einziges Kriterium gibt, welches die Qualität von Software misst.
Vielmehr ist es eine Kombination verschiedener Kriterien.
Auf diese wird im folgenden Abschnitt eingegangen.
TODO verweis uf übernöchsti abschnitt und so.

\subsection{Kriterien}
In den folgenden Abschnitten werden die Kriterien erklärt, welche nach \citeauthor{hoffmann2013software} relevant sind für die Softwarequalität.
Die Kriterien ersten vier Kriterien sind dabei für den Kunden wichtig, da diese direkte Auswirkung auf ihn haben.
Die weiteren Kriterien sind für den Hersteller der Software relevant.
\subsubsection{Funktionalität}
Die Funktionalität gibt an, ob die spezifizierten Anforderungen erfüllt sind.
Funktionale Fehler werden meist durch Bug in der Implementierung verursacht, können ihren Ursprung jedoch auch in fehlenden oder falsche verstandenen Spezifikationen haben.
Sie können durch den Einsatz von Software-Qualitätsicherungstools vorgebeugt werden.

\subsubsection{Performance}
Mit Performance sind die Anforderungen an die Software während deren Laufzeit gemeint.
Für einfache Desktop Anwendungen stellt dieses Kriterium meist kein Problem dar.
Handelt es sich bei der Anwendungen jedoch um ein Echtzeitsystem, so ist die Performance eines der wichtigsten Kriterien.

\subsubsection{Zuverlässigkeit}
Mit diesem Kriterium ist gemein, wie zuverlässig eine Software ihre Funktionen ausführt.
Kommt es oft zu Fehlern oder Abstürzen, so ist die Zuverlässigkeit tief.
Sie ist stark an die anderen Kriterien gekoppelt.
Hat die Software inkorrekte Funktionalität oder schlechte Performance, so ist auch Zuverlässigkeit tief.

\subsubsection{Benutzbarkeit}
Die Eingenschaften einer Software, welche mit dem Menschen interagieren, sind in diesem Kriterium zusammengefasst.
Im Abschnitt \ref{usability} wird diese Thematik vertieft.

\subsubsection{Wartbarkeit}
Um an einer Software auch nach der ersten Inbetriebnahme weiter zu entwickeln, so muss diese Wartbar sein.
Es soll möglich sein, erkannte Bugs einfach zu beheben.
Die Software soll so aufgebaut sein, dass sie nicht vollständig umgebaut werden muss, nur um eine neue Funktion hinzuzufügen.


\subsubsection{Transparenz}
Mit diesem Kriterium wird bewertet, wie transparent das Program intern umgesetzt ist.
Alle Teilkomponenten der Software einfach zu verstehen sein.
Tendenziell verschlechtert sich die Transparenz im verlaufe der Weiterentwicklungen.


\subsubsection{Übertragbarkeit}
Die Übertragbarkeit gibt an, ob sich eine bestehende Software in eine andere Umgebung übertragen lässt.
Kann ein Programm nur auf einer bestimmten Betriebsstemmversion oder gar nur auf einem einzigen Rechner ausgeführt werden, so ist dessen Übertragbarkeit sehr schlecht.
Mit Umgebung ist jedoch nicht nur die technische Umgebung wie das Betriebssystem gemeint sonder sie kann auch Aspekte wie die Sprache oder Kultur beinhalten.

\subsubsection{Testbarkeit}
Software ist meist so komplex, dass es nicht ausreicht lediglich die Benutzerschnittstellen zu testen.
Die möglichen Kombinationen von Eingabeparametern sind dabei viel zu umfangreich.
So müssen einzelne Komponenten einzel getestet werden.
Dies ist nur möglich, wenn diese so entwickelt werden, dass sie auch testbar sind.
Damit ist gemeint, dass bspw. ein Algorithmus so implementiert ist, dass er von keinerlei internen Zustände der Anwendungen abhängig ist. 


\subsection{Produktqualität}

\subsubsection{Konstruktive Qualitätssicherung}
typing
fehlertolerante programmierung
dokumentation
portabilität
\subsubsection{Analytische Qualitätssicherung}
test, arten von tests, metriken

static analyiss
verifikation


\subsection{Prozessqualität}
Versionsverwaltung
CI/CD
Vorgehensmodelle



\subsection{Software-Qualitätssicherung}
Unter Software-Qualitätssicherung versteht sich eine systematische und geplante Sammlung von Aktionen, 
welche die Sicherheit geben, dass die erstellte Software den Anforderungen entspricht.
Diese sollen den Entwicklungsprozess evaluieren und sicherstellen,
dass die Software im gegebenen zeitlichen sowie finanziellen Rahmen erstellt werden kann \parencite{galin2004software}. 

Sie unterscheidet sich von der Qualitätskontrolle indem sie nicht das fertige Produkt sondern den Herstellungsprozess evaluiert und prüft \parencite{galin2004software}.


\subsection{SonarQube}\label{quality:sonar}
SonarQube ist eine Software-Qualitatssicherungstool, welches Code analysiert und Berichte zur Codequalität erstellt.
Die Berichte könne beispielsweise Code-Style-Verletzungen, Designfehler oder gar Sicherheitslücken aufzeigen.
Da die Analysen statisch sowie dynamisch durchgeführt werden, können auch Metriken zu Codeabdeckung durch Tests erstellt werden \parencite{malloy_2021}.
SonarQube berichtet nicht nur über erkannte Probleme sondern beinhaltete auch Funktionen um diese zu Verwalten.
Ein Problem kann beispielsweise direkt der Person zugewiesen werden, welche es bearbeiten soll.
Meldungen, welche nicht bearbeitet werden, können entsprechend markiert werden, so dass sie in Zukunft nicht erneut auftreten.
Da SonarQube die Historie der Berichte speichert, werden neue Probleme speziell hervorgehoben.
Wie sich die totale Anzahl der Problem über die Zeit entwickelt, wird ebenfalls angezeigt und gibt einen Trend der Softwarequalität an.

% Maintainability, Security, Reliability, Complexity

\subsubsection{Funktionsweise}\label{sonar:funktionsweise}
In einem produktiven Umfeld ist die Funktionsweise von SonarQube wie folgt:
SonarQube wird auf einem Server installiert und mit einem \ac{CI} Server verbunden.
Wenn Entwickler Änderungen am Code in das jeweilige \ac{SCM} System pushen löst dies einen Build auf dem \ac{CI} Server aus.
Dieser Build beinhaltet Sonar Scanner.
Ist der Build abgeschlossen werden die Berichte der Scanner an die SonarQube Instanz übermittelt.
Dort werden sie verarbeitet, in der Datenbank abgelegt und über eine Benutzerschnittstelle dargestellt \parencite{malloy_2021}.

\subsection{Qualitätssicherungstools bei der Landis+Gyr}
Für die Qualitätssicherung der Firmware setzt die Landis+Gyr aktuell C-STAT\footnote{https://www.iar.com/cstat} für statische Codeanalysen ein.
Eine kürzlich durchgeführte Evaluation hat jedoch ergeben, dass SonarQube die geeignetere Lösung wäre.
Dieser Umstieg ist zum Zeitpunkt dieser Arbeit bereits geplant, jedoch noch nicht realisiert.
Bei den verschiedenen C\# Projekten, welche in diesem Kapitel genannt wurden, werden keinerlei Qualitätsicherungstools eingesetzt.
