% https://dvmarcilio.github.io/papers/icpc2019.pdf
% http://aagasc.edu.in/cs/books/Software%20Quality%20Assurance%20From%20Theory%20to%20Implementation.pdf
% see document in Downloads!

\section{Softwarequalität}
Die \ac{IEEE} definiert Softwarequalität als den Grad, zu dem ein System,
eine Komponente oder ein Prozess die spezifizierten Anforderungen sowie die Erwartungen der Benutzer erfüllt.
Des weiteren definiert sie, dass sich Software aus vier Komponenten zusammensetzt.
\begin{itemize}
   \item Das Program, welches aus Code besteht und vom Computer ausgeführt wird.
   \item Prozeduren, welche definieren, wie und von wem das Program ausgeführt wird.
   \item Dokumentation für Entwickler, Benutzer und Betreiber des Programs.
   \item Die Daten, welche benötigt werden, um das Programm zu betreiben.
\end{itemize}
Ist die Qualität dieser vier Komponenten hoch und erfüllen die Anforderungen und Erwartungen, so ist auch die Qualität der Software hoch \parencite{galin2004software}.
Wie dies erreicht werden kann, wird im folgenden Abschnitt beschrieben.

\subsection{Sicherstellung von Softwarequalität}



\subsection{Softwarefehler}
Nicht jeder Fehler führt zu einem Versagen