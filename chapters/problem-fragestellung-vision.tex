\chapter{Problem, Fragestellung, Vision}

% Aus Aufbau_Bericht.pdf
% Welche Ziele, Fragestellungen werden mit dem Projekt verfolgt? Die Bedeutung, Auswirkung und
% Relevanz dieses Projektes für die unterschiedlichen Beteiligten soll aufgeführt werden.
% Typischerweise wird hier ein Verweis auf die Aufgabenstellung im Anhang gemacht.

Die Aufgabenstellung dieses Projektes, welche im Anhang \ref{anhang:aufgabenstellung} zu finden ist, beschreibt die Ausgangslage dieses Projektes wie folgt:

\dq Entwickler, welche Stromzähler programmieren, müssen für Debugging und Testing manuell Objekte aus dem Stromzähler auslesen.
Dies ist über das DLMS Protokoll möglich.
Aktuell wird bei der Landis+Gyr dazu das Tool \textit{DMT2} verwendet.
\textit{DMT2} ist darauf spezialisiert, mittels Scripts automatisierte Abfragen auszuführen.
Zusätzlich bietet es eine Funktion für das manuelle Lesen und Schreiben einzelner Objekte an.
Diese nennt sich \textit{Quick Access} und ist für den Benutzer umständlich zu bedienen und das Interface ist nicht mehr zeitgemäss.
Im Rahmen dieser Arbeit soll ein neues Tool entwickelt werden, welches den \textit{Quick Access} ablöst und somit die täglichen Entwicklungsarbeiten erleichtert.\dq

\section{Erwartete Resultate}\label{erwarteteResultate}
Wie oben erwähnt, soll eine Softwarelösung entwickelt werden, welche die bestehende ablöst.
Diese soll modern und benutzerfreundlich sein.
Nebst den Funktionalitäten welche bereits in der bestehenden Software vorhanden sind, sollen deren Benutzer befragt und gewünschte Zusatzfunktionen implementiert werden.
Wie diese Applikation technisch umgesetzt wird und optisch daherkommt, lässt die Aufgabenstellung offen.
Es wird jedoch gefordert, dass der gewählte Technologie Stack für die Landis+Gyr günstig zu unterhalten sein soll.

Des Weiteren sollen die Stärken und Schwächen der beiden Programme analysiert und dokumentiert werden.
Von den Benutzern gewünschten Funktionen, welche im Rahmen dieses Projektes nicht umgesetzt werden konnten, sollen ebenfalls festgehalten werden.

\section{Weitere Vorgaben}
TODO Scrum und ander Prozesse erwähnen

\subsection{Schwerpunkte}
Während der Bearbeitung des Auftrages sollen folgende Fragen bearbeitet werden:
\begin{itemize}
   \item Wie kann eine hohe Softwarequalität sichergestellt werden?
   \item Was macht eine ergonomische Benutzerschnittstelle aus?
\end{itemize}