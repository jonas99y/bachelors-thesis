\chapter{Problem, Fragestellung, Vision}

Die Landis+Gyr AG prüfte intern mögliche Projekte für eine Bachelorarbeit im Bereich Machine Learning.
Eine dieser Ideen war das Erkennen einzelner Stromverbraucher, wie beispielsweise des Backofens oder der Waschmaschine,
basierend auf den Messwerten des gesamten Haushalts.
Bei dieser Idee sowie auch bei anderen traten folgende Probleme auf:
\begin{itemize}
      \item Für Supervised Learning (TODO Quelle), müssen Daten vorhanden sein,
            welche über Labels verfügen (TODO mehr dazu Kapitel XY).
            Solche Daten waren nicht verfügbar.
      \item Die gemessenen Stromzählerdaten werden meist über 15 Minuten aggregiert und dann übermittelt.
            Diese Frequenz und Auflösung lässt kaum Echtzeit Analysen der Messdaten zu.

\end{itemize}

Um für die genannten Probleme Lösungen zu erarbeiten wurde eine Aufgabenstellung mit dem Titel ``Smart Metering into the cloud`` (siehe Anhang \ref{anhang:aufgabenstellung}) verfasst,
welche Ausgangspunkt dieser Arbeit ist.

\section{Zusammenfassung Aufgabenstellung}
\label{aufgabenstellung}

Es soll eine Platform entwickelt werden, welche Messdaten mittels \ac{MQTT} Protokoll  empfangen kann.
Diese Daten sollen abgespeichert werden und über eine Benutzerschnittstelle abrufbar sein.
Dort sollen sie visuell mittels Graphen dargestellt werden.
Über diese Schnittstelle soll es eine Möglichkeit geben, den Daten Labels hinzuzufügen,
damit diese für Machine Learning verwendet werden können.
Es wird verlangt, dass die Lösung mittels Container umgesetzt wird und dass die Benutzerschnittstelle für mobile Geräte entwickelt wird.
Bei den restlichen Technologien verfügt das Projektteam über kreativen Freiraum.
Die konkreten Technologien werden im Kapitel \ref{p1:evaluation_tech} evaluiert.
Dabei soll agil vorgegangen werden um Änderungen der Anforderungen im Verlauf des Projekts zu ermöglichen.
Wie dieses Vorgehen umgesetzt wurde, ist im Kapitel \ref{methoden} beschrieben.
Die vollständige Aufgabenstellung ist dieser Arbeit als Anhang \ref{anhang:aufgabenstellung} beigelegt.

% Aus Aufbau_Bericht.pdf
% Welche Ziele, Fragestellungen werden mit dem Projekt verfolgt? Die Bedeutung, Auswirkung und
% Relevanz dieses Projektes für die unterschiedlichen Beteiligten soll aufgeführt werden.
% Typischerweise wird hier ein Verweis auf die Aufgabenstellung im Anhang gemacht.