\chapter{Ausblick}

Stromzähler mit MQTT + Tool deploy in cloud + label label label
..
ML/AI

% Ned ganz sicher ob die persönliche Reflexione bi "Ausblick" korrekt sind...
\section{Reflexion}
\subsection{Jonas}
Positiv: Python, FastAPI, Podman etc.

Negativ:
React

\subsection{Benjamin}

Grundsätzlich bin ich sehr zufrieden mit dem erreichten Ergebnis.
Persönlich finde ich eine gute Architektur zu entwickeln die sowohl simpel
als auch im nötigen masse erweiterbar ist eine grosse Herausforderung.
Aus meiner Erfahrung wird eine Architektur oft komplizierter als nötig.
Genau aus diesem Grund bin ich sehr zufrieden mit der entstandenen Architektur.
Sie ist nur so kompliziert wie nötig und die Komplexität dient dem Zweck
der Automatisierung was bei der Entwicklung und dem Deployment wieder einfacher wird.

Zudem war es für mich das erste Mal, dass ich ein richtiges Multicontainerprojekt
mit podman aufgesetzt habe. Mit den in diesem Projekt gemachten Erfahrungen
würde ich jederzeit wieder podman anstatt docker-compose für mein Setup wählen.

Auch die Python Library poetry, sqlmodel und FastAPI habe ich im Kontext dieses
Projektes zum ersten mal benutzt. Zwar hab ich schon viel Erfahrung mit
ähnlichen Libraries, es ist aber eine willkommene Abwechslung bei einem
Greenfield Projekt neue Technologien auszuprobieren und zu sehen welche Probleme sie lösen
und ob sie wirklich besser sind als ihre Vorgänger.

Nicht zuletzt war es für mich ein spannender Einblick in die Welt von Landis+Gyr
und Stromzähler. Ich durfte viele Sachen in dieser Domäne im Verlaufe des Projektes lernen.

Gerne hätte ich noch etwas Zeit ins Frontend investiert, dies war aber aus zeitlichen Gründen
nicht mehr möglich.

\section{Ungelöste Probleme}

\subsection{\ac{TLS} Client Zertifikat}

Zwar werden die übertragenen Daten zurzeit mit \ac{TLS} verschlüsselt aber
momentan kann jeder, der das \ac{MQTT} Broker Passwort kennt Daten an den
Dataingress senden. Dieses Passwort soll natürlich nicht öffentlich sein.
Das Problem ist jedoch, durch Auslesen des Speichers der Stromzähler
könnte das Passwort herausgefunden werden.
Zwar könnte man ein Passwort pro gerät generieren, eine bessere Möglichkeit
wäre jedoch die Verwendung von Client Zertifikaten \parencite{rfc5246_2021}.s
Dadurch können nicht nur die Stromzähler die Integrität des \ac{MQTT} Brokers
verifizieren sondern auch der Broker ob nur die richtigen Geräte
Daten senden.



\section{Weitere Ideen}
THD: Festlegen für schwellen von Werten, dass Warnungen generiert werden können.
THD thematik nicht weiterverfolgt, da Auftraggeber keine Daten hatte.

Mithilfe der entwickelten Platform ist es nun möglich, Messdaten in hoher Auflösung zu speichern
und mit Labels zu versehen.
Somit kann die Idee des Machine Learning Projekts, welche im Kapitel \ref{cap1} Problem, Fragestellung, Vision erwähnt wurde und Ursprung dieser
Arbeit war, neu beurteilt werden.


% Aus Aufbau_Bericht.pdf
% Reflexion der eigenen Arbeit, ungelöste Probleme, weitere Ideen.