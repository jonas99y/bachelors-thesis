\chapter{Ausblick}

Stromzähler mit MQTT + Tool deploy in cloud + label label label
..
ML/AI

% Ned ganz sicher ob die persönliche Reflexione bi "Ausblick" korrekt sind...
\section{Reflexion}
\subsection{Jonas}
Positiv: Python, FastAPI, Podman etc.

Negativ:
React

\subsection{Benjamin}
\section{Ungelöste Probleme}



\section{Weitere Ideen}
THD: Festlegen für schwellen von Werten, dass Warnungen generiert werden können.
THD thematik nicht weiterverfolgt, da Auftraggeber keine Daten hatte.

Mithilfe der entwickelten Platform ist es nun möglich, Messdaten in hoher Auflösung zu speichern
und mit Labels zu versehen. Somit kann die Idee (TODO verweis kapitel 1), welcher der Ursprung dieser
Arbeit war, neu beurteilt werden.

\subsection{\ac{TLS} Client Zertifikat}

Zwar werden die übertragenen Daten zurzeit mit \ac{TLS} verschlüsselt aber
momentan kann jeder, der das \ac{MQTT} Broker Passwort kennt Daten an den
Dataingress senden. Dieses Passwort soll natürlich nicht öffentlich sein.
Das Problem ist jedoch, durch Auslesen des Speichers der Stromzähler
könnte das Passwort herausgefunden werden.
Zwar könnte man ein Passwort pro gerät generieren, eine bessere Möglichkeit
wäre jedoch die Verwendung von Client Zertifikaten \parencite{rfc5246_2021}.
Dadurch können nicht nur die Stromzähler die Integrität des \ac{MQTT} Brokers
verifizieren sondern auch der Broker ob nur die richtigen Geräte
Daten senden.


% Aus Aufbau_Bericht.pdf
% Reflexion der eigenen Arbeit, ungelöste Probleme, weitere Ideen.