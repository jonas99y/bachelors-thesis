\chapter{Ausblick}
Wie bereits im Kaptiel \ref{cap1} beschrieben, dient dieses Projekt als Vorarbeit für mögliche
Machine Learning Projekte, welche gelabelte Daten benötigen.
Diese gelabelten Daten können mithilfe der erarbeiten Applikation erstellt werden.
Die im Anhang \ref{anhang:readme-main} bis \ref{anhang:readme-tests} aufgeführten Readme Files dienen dazu als Einstiegspunkt
um die Applikation in Betrieb zu nehmen.



\section{Ungelöste Probleme}

\subsection{\ac{TLS} Client Zertifikat}

Zwar werden die übertragenen Daten zurzeit mit \ac{TLS} verschlüsselt aber
momentan kann jeder, der das \ac{MQTT} Broker Passwort kennt Daten an den
Dataingress senden. Dieses Passwort soll natürlich nicht öffentlich sein.
Das Problem ist jedoch, durch Auslesen des Speichers der Stromzähler
könnte das Passwort herausgefunden werden.
Zwar könnte man ein Passwort pro gerät generieren, eine bessere Möglichkeit
wäre jedoch die Verwendung von Client Zertifikaten \parencite{rfc5246_2021}.s
Dadurch können nicht nur die Stromzähler die Integrität des \ac{MQTT} Brokers
verifizieren sondern auch der Broker ob nur die richtigen Geräte
Daten senden.

\subsection{Benutzerauthentifizierung}
Da im Auftrag nicht gefordert und für die Entwicklungsarbeiten nur hinderlich,
wurde auf das Authentifizieren von Benutzer bei Zugriffen auf die \ac{API} vollständig verzichtet
Bei der aktuellen Implementation sind alle Daten frei zugänglich.
Wir die Applikation also mit echten Daten betrieben, so müsste sichergestellt werden, dass ein Zugriff durch unbefugte
Personen nicht möglich ist.



\section{Weitere Ideen}
Während des Projektes entstand beim Auftraggeber die Anforderung, dass \ac{THD} von der Applikation
verarbeitet werden könne sollen. Diese Werte können frühzeitige Indikatoren für Fehler in Kraftwerken sein.
Dabei entstand die Idee, dass diese Werte in der Applikation analysiert
und im Fall von Anomalien entsprechende Warnungen generiert werden könnten.

THD: Festlegen für schwellen von Werten, dass Warnungen generiert werden können.
THD thematik nicht weiterverfolgt, da Auftraggeber keine Daten hatte.

Mithilfe der entwickelten Platform ist es nun möglich, Messdaten in hoher Auflösung zu speichern
und mit Labels zu versehen.
Somit kann die Idee des Machine Learning Projekts, welche im Kapitel \ref{cap1} Problem, Fragestellung, Vision erwähnt wurde und Ursprung dieser
Arbeit war, neu beurteilt werden.


% Aus Aufbau_Bericht.pdf
% Reflexion der eigenen Arbeit, ungelöste Probleme, weitere Ideen.