% Reflexion der eigenen Arbeit, ungelöste Probleme, weitere Ideen.

\chapter{Ausblick}

\section{ATS aufsplitten}\label{ausblick:ats_split}
Einzelnen Komponenten wie bswp. DLMS kommunikation als NuGet Paket bereitstellen.


\section{Class Descriptions}
Wie werden dies aktuell gehalten?


\subsection{Testagent mit Zähler}
Im Abschnitt \ref{Integrationstests} wurde beschrieben, dass für das ausführen der Integrationstests ein angeschlossener Stromzähler vorausgesetzt wird.
TODO

\subsection{SonarQube}
In Abschnitt \ref{s6:sonar} wurde erklärt, wieso im Rahmen dieser Arbeit lediglich eine lokale Instanz des Qualitätssicherungstools SonarQube eingesetzt wurde.
Für den Unterhalt und die Weiterentwicklung der Anwendung sollte jedoch eine Instanz auf einem Server eingesetzt werden, welche mit dem \ac{CI} Server verbunden ist.
Dies hätte zwei Vorteile:
\begin{itemize}
   \item Die Berichte zu Codequalität werden bei jeder Änderung des Codes automatisch erstellt. 
Entwickler müssen sich nicht selber darum kümmern.
   \item Die Metriken zur Qualität der Anwendung sind für alle Personen einsehbar.
\end{itemize}

\subsection{Zertifikat für die Signierung der Anwendung}\label{ausblick:cert}