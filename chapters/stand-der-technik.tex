\chapter{Stand der Technik}

Das folgende Kapitel gibt einen Einblick in den Stand der Technik und Hingergrundinformationen
für diese Arbeit.

\section{Einführung}

TODO: Jonas, Stand der stromzähler Technik.

Ein Stromzähler \cite{wikipedia:meter} misst den Stromverbrauch eines Haushaltes über eine
Zeitspanne. Diese Daten können sowohl für den Endverbraucher als auch für Kraftwerke
und Verteiler interessant sein. Beispielsweise können \ac{THD} \cite{siemens:disw_2021}
Werte eine Aussage über den Zustand eines Generators in einem Kraftwerk machen. \cite{thd-generators}

Moderne Stromzähler verfügen über zu wenig Rechenleistung um die Daten lokal
aufbereiten zu können. \cite{dc450-technical-data}
Aus diesem Grund sollen die Daten via \ac{MQTT} \footnote{https://mqtt.org/} Protokoll
an einen Server zur Weiterverarbeitung gesendet werden.

\ac{MQTT} eignet sich vor allem für \ac{IoT} Geräte mit geringer Leistungsfähigkeit
und wurde vom Auftraggeber bereits so vorgegeben, da Sie von verschiedenen
Stromzählern bereits unterstützt wird.

\section{Hosting und Deployment}

Zu beginn des Projektes war die Idee, das Projekt in der Google Cloud zu hosten.
Da jedoch die Infrastruktur auf Seiten des Auftraggebers noch nicht dazu bereit
war, konnte dies nicht umgesetzt werden. Stadtdessen soll eine möglichst
generische lösung, welche vielleicht später auf einen Kubernetes\footnote{https://kubernetes.io/}
Cluseter deployen zu können. Damit ein Produkt reproduzierbar
deployt werden kann, werden heutzutage Container verwendet. \cite{what-is-a-container}

Das wohl bekannteste und weit verbreitetste Container Framework
ist Docker\footnote{https://www.docker.com/}. Docker hat jedoch einige Nachteile.
So ist Docker beispielsweise per default nicht rootless.\cite{docker:rootless}
Das bedeutet, dass ein in einem Container gestarteter Prozess als root user
auch auf dem Host system unter der gleichen Benutzer läuft. Wenn jetzt also
der Prozess aus der Containerisolation ausbricht\footnote{Dies kann beispielsweise durcheine Sicherheitslücke passieren}
ist er auch auf dem Hostsystem root.\cite{so_2020}
Zudem ist Docker an einen Damon gebunden, der im Hintergrund läuft.
Dadurch werden alle Container neu gestartet wenn der Daemon neu gestartet werden muss.\cite{docker:daemon}
Aus diesen gründen setzen viele Projekte podman\footnote{https://docs.podman.io/en/latest/}
als Container Engine ein. Podman ist rootless by default und erlaubt Daemonless Container.
Zudem können multicontainer Projekte einfach mit Kubernetes Config files
erstellt werden.\cite{redhat:podman-pods}


