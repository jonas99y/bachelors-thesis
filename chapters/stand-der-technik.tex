\chapter{Stand der Technik}
% Bezogen auf die eigenen Zielsetzungen und Fragestellungen soll aufgezeigt werden, wie andere
% dieses oder ähnliche Probleme gelöst haben. Worauf können Sie aufbauen, was müssen Sie neu
% angehen? Wodurch unterscheidet sich Ihre Lösung von anderen Lösungen? Für wissenschaftlich
% orientierte Arbeiten sei hier explizit auf (Balzert, S. 66 ff) verwiesen.
In diesem Kapitel werden gängige Technologien vorgestellt, welche für die Lösung der Aufgabenstellung relevant sind.
Des Weiteren wir Theorie zu den Themen Softwarequalität und Benutzerschnittstellenergonomie aufgearbeitet.


\section{DLMS}
\ac{DLMS} ist eine Sammlung von offenen Standards, welche ein Applikations- und ein Transferprotokoll definieren.
Die Protokolle basieren auf dem \ac{OSI} Modell, sind jedoch auf die Layer \textit{physical}, \textit{data link}, \textit{transport} und \textit{application} reduziert.
Diese sind in Abbildung \ref{fig:dlmsOsi} dargestellt.
Dabei ist ersichtlich, dass auf den unteren Layern unterschiedliche Kommunikationskanäle verwendet werden können.

% TODO image ist aus PDF. Quelle angeben?
\begin{figure}[H]
   \centering
   \includegraphics[width=0.7\textwidth]{gfx/Dlms_osi.png}
   \caption{
       DLMS/COSEM Kommunikationslayer
   }
   \source{\cite{vyas2012advance}}
   \label{fig:dlmsOsi}
\end{figure}

Der Applicationlayer des \ac{DLMS} Protokolls definiert die Funktionen eines intelligenten Stromzählers als Objekte \parencite{vyas2012advance}.

\subsection{COSEM}\label{cosem}
Der \ac{COSEM} Standard, welcher teil von \ac{DLMS} ist, beschreibt eine Sammlung von logischen Geräten, welche gemeinsam in einem physischen Gerät untergebracht werden.
Diese logischen Geräte bestehen aus mehreren Attributen und Methoden, welche jeweils zu Objekten zusammengefasst sind.
Spezifische Objekte können mittels \ac{OBIS} Code adressiert werden.
Die Schnittstellen zu den Objekten sowie eine Liste von standard \ac{OBIS} Codes sind in \ac{COSEM} enthalten  \parencite{vyas2012advance}.


\section{Landis+Gyr intern}\label{lgintern}
In den folgenden Abschnitten werden Konzepte, Anwendungen und Produkte erklärt, welche bei der Landis+Gyr intern verwendet werden.

\subsection{Picasso Platform}\label{picasso}
Picasso ist der Name einer Software Platform der Landis+Gyr.
Mehrere Teams, verteilt auf vier Kontinente, arbeiten am C++ Code dieser Platform.
Sie beinhaltet grundlegende Komponenten wie beispielsweise das Betriebs- und Filesystem sowie Funktionen welche von allen Produkttypen verwendet werden.
Die Teams welche die konkreten Stromzähler-Produkte entwickeln, bauen auf Picasso auf und ergänzen die Platform um Funktionen, welche nur von ihrem Produkt verwendet werden.
Damit die Platform mit unterschiedlichen Konfigurationen getestet werden kann, verfügt sie über mehrere Referenzprodukte.
Referenzprodukte sind Produkte, welche nicht Verkauft werden, sondern nur für interne Zwecke verwendet werden.
Am Standort Cham arbeitet ein Team an der Platform sowie eines an einem Produkt, dem Stromzähler Model \textit{E660}.
Der Autor dieser Arbeit ist seit mehreren Jahren Teil des Platformteams.



\subsection{Object Model und Class Description}\label{objectModelsClassDescriptions}
Im Absatz \ref{cosem} wurde erwähnt, dass die Funktionen eines intelligenten Stromzählers mittels Objekten abstrahiert werden.
Bei der Landis+Gyr werden alle Objekte eines bestimmten Produkts jeweils in einem Object Model zusammengefasst.
Anhand des Object Models entwerfen Architekten die Funktionen eines Zählers.
Sie verwenden eine Excel Tabellen um die Werte der Object Models zu bearbeiten.
In Abbildung \ref{fig:objectModel} ein der Header sowie ein Objekt aus einer solchen Tabelle dargestellt.

\begin{figure}[H]
   \centering
   \includegraphics[width=1.0\textwidth]{gfx/objectModel.png}
   \caption{
       Ausschnitt aus der Excel Tabelle eines Object Models
   }
   \label{fig:objectModel}
\end{figure}

Diese Objekte sind jeweils Instanzen von \ac{COSEM} Klassen.
Die Funktionalität und Schnittstelle dieser Klassen sind in sogenannten Class Descriptions dokumentiert.
Jede Class Description ist in einem eigenen Word Dokument abgelegt.
Ein Beispiel dazu in in Abbildung \ref{fig:classDescription} ersichtlich.

\begin{figure}[H]
   \centering
   \includegraphics[width=1.0\textwidth]{gfx/ClassDescription.png}
   \caption{
       Ausschnitt aus einer Class Description
   }
   \label{fig:classDescription}
\end{figure}

Das Object Model in Kombination mit den verwendeten Class Descriptions dient bei der Firmwareentwicklung als Referenz für die Implementation der \ac{DLMS}/\ac{COSEM} Schnittstelle.
Um die Integrität dieser Dateien sicherzustellen und um diese in andere Dateiformate zu konvertieren, entwickelt und betreibt die Landis+Gyr mehrere Programme.
Diese werden im nächsten Abschnitt erläutert.

\subsubsection{Tools}
Das Programm \textit{Picasso Tools} stellt ein Plugin für Excel bereit, welches dem Benutzer ermöglicht den Inhalt eines Object Models zu validieren.
Dabei werden die Objekte anhand von vordefinierten Regeln auf Korrektheit und Vollständigkeit überprüft.
Des Weiteren wird mithilfe der entsprechenden Class Description überprüft, ob die Attribute und Methoden eines Objekts korrekt sind.
Die Class Descriptions müssen dazu als XML Dateien vorhanden sein.
Da die Class Descriptions wie zuvor beschrieben jedoch Word Dokumente sind, müssen diese mit dem Programm \textit{Description Tools} exportiert werden.
Analog zu \textit{Picasso Tools} stellt \textit{Description Tools} ein Plugin, in diesem Fall für Word, bereit, welches Dokumente validieren und als XML exportieren kann.

Nebst den Plugins bieten beide Programme auch ein \ac{CLI} an.
Dieses wird von einem Server verwendet, welcher jeden Tag alle Object Models und Class Description validiert und als XML exportiert.
Anhand der Validierungsresultate erstellt er einen Report, welche einen Überblick über die Qualität der Dokumente gibt sowie auf Fehler hinweist.

Beide Tools sind in der Programmiersprache C\# entwickelt und nutzen die Komponente \textit{InfraLib}.
Diese ist eine C\# Bibliothek, welche Funktionalitäten wie das Parsen der XML files anbietet. 
\footnote{\textit{Picasso Tools}, \textit{Description Tools} sowie \textit{InfraLib} wurden vom Autor dieser Arbeit erstellt, als dieser seine Berufslehre bei der Landis+Gyr absolvierte und seither weiterentwickelt }

\subsection{Firmwareentwicklung}\label{fwEntwicklung}
Am Standort Cham entwickelt die Landis+Gyr Firmware für intelligent Stromzähler.
Dazu wird die Programmiersprache C++ eingesetzt.
Wie im Abschnitt \ref{objectModelsClassDescriptions} bereits erwähnt, stützt sich die Firmware auf Object Models.
Diese werden verwendet, um jenen Code zu generieren, welcher die \ac{COSEM} Objekte in der Firmware instanziert.
Ein Tool, welches in der Programmiersprache Python geschrieben ist, parst die XML Repräsentation des Object Models und mit Hilfe von Templates den entsprechenden Code.
Da das verwendete Buildsystem, SCons \footnote{https://scons.org/}, ebenfalls mit Python arbeitet ist Python hinter C++ die zweit meist verwendete Sprache im Projekt der Firmwareentwicklung.

\subsection{E66C Testing}\label{pythonTesting}
\textit{E66C} ist die Modellbezeichnung eines Kommunikationsmoduls, welches von der Landis+Gyr in Cham entwickelt wird.
Dabei handelt es sich um eine Komponente, welcher auf gewisse Stromzählermodelle aufgesteck werden kann und diesen um Kommunikationsfunktionen erweitert.

\begin{figure}[H]
   \centering
   \includegraphics[width=0.7\textwidth]{gfx/landis-e66c.jpg}
   \caption{
      Kommunikationsmodule E66C der Landis+Gyr
   }
   \source{Landis+Gyr AG}
   \label{fig:e66c}
\end{figure}

Um das Zusammenspiel des \textit{E66C} mit einem Stromzähler automatisiert zu testen, entwickelte das Team in Cham das Programm \textit{libpydlms}.
Dieses ermöglicht das Lesen und Schrieben von \ac{COSEM} Objekten in der Sprache Python.


\subsection{ATS}\label{ats}
\ac{ATS} ist eine Software welche für das Testen von intelligenten Stromzählern verwendet wird.
Dazu werden Testscripts in einer \ac{ATS} spezifischen Scriptsprache benötigt.
Über diese Scripts könne Werte des Zählers über die \ac{DLMS} Schnittstelle geschrieben, ausgelesen und mit Erwartungswerten vergleichen werden.
Zusätzlich ist die Steuerung der Testumgebung möglich.
Befindet sich der Testaufbau beispielsweise auf dem Tisch des Entwicklers, so wird der Stromzähler in der Regel nicht über eine richtige Stromquelle betrieben sondern an einen Emulator angeschlossen, welcher diese ersetzt.
In diesem Fall wird der Emulator über das \ac{ATS} Testscript gesteuert.

Die \ac{ATS} wurde von der Landis+Gyr in der Programmiersprache C\# entwickelt und wird aktuell für das Testen aller Stromzähler der Picasso Platform (siehe \ref{picasso}) eingesetzt.



\subsection{DMT2}\label{dmt}
Die Software \ac{DMT2} wurde für die Landis+Gyr von einem externen Lieferanten entwickelt.
Mit ihr können Scripts geschrieben und ausgeführt werden, welche mittels \ac{DLMS} mit intelligenten Stromzähler kommunizieren.
Diese Scripts können um einfach Benutzeroberflächen erweitert werden, welche mit XML deklarativ definiert werden.
Verwendet werden diese, um Interaktionen, welche die Entwickler oft durchführen müssen, zu vereinfachen.
In Abbildung \ref{fig:dmt2logger} wir beispielsweise ein Script und die dazugehörige Benutzerschnittstelle für die Laufzeitkonfiguration des Loggers eines Zählers gezeigt.
Wenn dort beispielsweise das Log Level verändert wird, schreibt das Script den entsprechenden Wert automatisch in die korrekte \ac{COSEM} Klasse.
\begin{figure}[H]
   \centering
   \includegraphics[width=1.0\textwidth]{gfx/dmt2logger.png}
   \caption{
      Ausschnitt aus DMT2 mit geladenem Logger-Konfigurationsscript
   }
   \label{fig:dmt2logger}
\end{figure}

Eine weitere Funktion des \ac{DMT2} ist \textit{Quick Access}.
Diese ermöglicht das Lesen und Schrieben einzelne Attribute sowie das ausführen von Methoden.
Das Ziel dieser Arbeit ist es, die Funktionalität des \textit{Quick Access} durch eine neue Softwarelösung zu ersetzten.
Im folgenden Abschnitt werden einige Stärken und Schwächen des \ac{DMT2} aufgezeigt.
Diese basieren auf einer Umfrage, welche unter den Benutzern des \ac{DMT2} durchgeführt wurde.
Im Abschnitt \ref{survey} ist mehr zur Umfrage zu lesen.

\subsubsection{Stärken}
\begin{itemize}
   \item Die Anwendung läuft stabil, Abstürze kommen sehr selten vor.
   \item Es kann zwischen verschiedenen Kommunikationseinstellungen gewechselt werden. Dazu muss jeweils eine entsprechende Datei geladen werden.
\end{itemize}

\subsubsection{Schwächen}
\begin{itemize}
   \item Fehlermeldungen werden als Dialog angezeigt und müssen jeweils bestätigt werden um fortzufahren.
   \item Es ist schwierig, nach spezifischen Objekten zu suchen.
   \item Das ausführen von Methoden und Schrieben von Attributen ist umständlich. Die Parameter müssen in einer XML Struktur eingetragen werden, welche anfällig für Fehler ist.
   \item Es wird jeweils nur die Antwort des zuletzt ausgeführten Befehls angezeigt.
   \item Es ist Umständlich mehrere Instanzen des \ac{DMT2} gleichzeitig zu nutzen. 
\end{itemize}


\section{WinUI3}
WinUI3 ist eine Platform, welche es erlaubt native Benutzeroberflächen für Windows zu entwickeln.
Sie ist teil der Windows APP SDK \footnote{https://docs.microsoft.com/en-us/windows/apps/windows-app-sdk/}.
Der Quellcode dazu ist Open Source \parencite{winuiintro}.
WinUI3 wird Microsoft als beste Technologie zur Erstellung von Benutzerschnittstellen bezeichnet und löst ältere Technologien wie \ac{WPF} oder \ac{UWP} ab.
Die erste stabile Version ist seit November 2021 verfügbar.
Neue Updates sollen regelmässig veröffentlich werden.
WinUI3 kann in Kombination mit dem .Net Framework von Microsoft verwendet werden, ist jedoch nicht davon abhängig.
Es werden die Programmiersprachen C\# und C++ unterstützt \parencite{winuiroadmap}.


% https://dvmarcilio.github.io/papers/icpc2019.pdf
% http://aagasc.edu.in/cs/books/Software%20Quality%20Assurance%20From%20Theory%20to%20Implementation.pdf
% see document in Downloads!

\section{Softwarequalität}\label{softwarequality}
In der DIN-ISO-Norm 9126 wird Software Qualität so definiert:
\dq Software-Qualität ist die Gesamtheit der Merkmale und Merkmalswerte eines Software-Produkts, die sich auf dessen Eignung beziehen, festgelegte Erfordernisse zu erfüllen.\dq

\citeauthor{hoffmann2013software} (\citeyear{hoffmann2013software}) hebt zu dieser Definition hevor, dass es nicht ein einziges Kriterium gibt, welches die Qualität von Software misst.
Vielmehr ist es eine Kombination verschiedener Kriterien.
Auf diese wird im folgenden Abschnitt eingegangen.
TODO verweis uf übernöchsti abschnitt und so.

\subsection{Kriterien}
In den folgenden Abschnitten werden die Kriterien erklärt, welche nach \citeauthor{hoffmann2013software} relevant sind für die Softwarequalität.
Die Kriterien ersten vier Kriterien sind dabei für den Kunden wichtig, da diese direkte Auswirkung auf ihn haben.
Die weiteren Kriterien sind für den Hersteller der Software relevant.
\subsubsection{Funktionalität}
Die Funktionalität gibt an, ob die spezifizierten Anforderungen erfüllt sind.
Funktionale Fehler werden meist durch Bug in der Implementierung verursacht, können ihren Ursprung jedoch auch in fehlenden oder falsche verstandenen Spezifikationen haben.
Sie können durch den Einsatz von Software-Qualitätsicherungstools vorgebeugt werden.

\subsubsection{Performance}
Mit Performance sind die Anforderungen an die Software während deren Laufzeit gemeint.
Für einfache Desktop Anwendungen stellt dieses Kriterium meist kein Problem dar.
Handelt es sich bei der Anwendungen jedoch um ein Echtzeitsystem, so ist die Performance eines der wichtigsten Kriterien.

\subsubsection{Zuverlässigkeit}
Mit diesem Kriterium ist gemein, wie zuverlässig eine Software ihre Funktionen ausführt.
Kommt es oft zu Fehlern oder Abstürzen, so ist die Zuverlässigkeit tief.
Sie ist stark an die anderen Kriterien gekoppelt.
Hat die Software inkorrekte Funktionalität oder schlechte Performance, so ist auch Zuverlässigkeit tief.

\subsubsection{Benutzbarkeit}
Die Eingenschaften einer Software, welche mit dem Menschen interagieren, sind in diesem Kriterium zusammengefasst.
Im Abschnitt \ref{usability} wird diese Thematik vertieft.

\subsubsection{Wartbarkeit}
Um an einer Software auch nach der ersten Inbetriebnahme weiter zu entwickeln, so muss diese Wartbar sein.
Es soll möglich sein, erkannte Bugs einfach zu beheben.
Die Software soll so aufgebaut sein, dass sie nicht vollständig umgebaut werden muss, nur um eine neue Funktion hinzuzufügen.


\subsubsection{Transparenz}
Mit diesem Kriterium wird bewertet, wie transparent das Program intern umgesetzt ist.
Alle Teilkomponenten der Software einfach zu verstehen sein.
Tendenziell verschlechtert sich die Transparenz im verlaufe der Weiterentwicklungen.


\subsubsection{Übertragbarkeit}
Die Übertragbarkeit gibt an, ob sich eine bestehende Software in eine andere Umgebung übertragen lässt.
Kann ein Programm nur auf einer bestimmten Betriebsstemmversion oder gar nur auf einem einzigen Rechner ausgeführt werden, so ist dessen Übertragbarkeit sehr schlecht.
Mit Umgebung ist jedoch nicht nur die technische Umgebung wie das Betriebssystem gemeint sonder sie kann auch Aspekte wie die Sprache oder Kultur beinhalten.

\subsubsection{Testbarkeit}
Software ist meist so komplex, dass es nicht ausreicht lediglich die Benutzerschnittstellen zu testen.
Die möglichen Kombinationen von Eingabeparametern sind dabei viel zu umfangreich.
So müssen einzelne Komponenten einzel getestet werden.
Dies ist nur möglich, wenn diese so entwickelt werden, dass sie auch testbar sind.
Damit ist gemeint, dass bspw. ein Algorithmus so implementiert ist, dass er von keinerlei internen Zustände der Anwendungen abhängig ist. 


\subsection{Produktqualität}

\subsubsection{Konstruktive Qualitätssicherung}
typing
fehlertolerante programmierung
dokumentation
portabilität
\subsubsection{Analytische Qualitätssicherung}
test, arten von tests, metriken

static analyiss
verifikation


\subsection{Prozessqualität}
Versionsverwaltung
CI/CD
Vorgehensmodelle



\subsection{Software-Qualitätssicherung}
Unter Software-Qualitätssicherung versteht sich eine systematische und geplante Sammlung von Aktionen, 
welche die Sicherheit geben, dass die erstellte Software den Anforderungen entspricht.
Diese sollen den Entwicklungsprozess evaluieren und sicherstellen,
dass die Software im gegebenen zeitlichen sowie finanziellen Rahmen erstellt werden kann \parencite{galin2004software}. 

Sie unterscheidet sich von der Qualitätskontrolle indem sie nicht das fertige Produkt sondern den Herstellungsprozess evaluiert und prüft \parencite{galin2004software}.


\subsection{SonarQube}\label{quality:sonar}
SonarQube ist eine Software-Qualitatssicherungstool, welches Code analysiert und Berichte zur Codequalität erstellt.
Die Berichte könne beispielsweise Code-Style-Verletzungen, Designfehler oder gar Sicherheitslücken aufzeigen.
Da die Analysen statisch sowie dynamisch durchgeführt werden, können auch Metriken zu Codeabdeckung durch Tests erstellt werden \parencite{malloy_2021}.
SonarQube berichtet nicht nur über erkannte Probleme sondern beinhaltete auch Funktionen um diese zu Verwalten.
Ein Problem kann beispielsweise direkt der Person zugewiesen werden, welche es bearbeiten soll.
Meldungen, welche nicht bearbeitet werden, können entsprechend markiert werden, so dass sie in Zukunft nicht erneut auftreten.
Da SonarQube die Historie der Berichte speichert, werden neue Probleme speziell hervorgehoben.
Wie sich die totale Anzahl der Problem über die Zeit entwickelt, wird ebenfalls angezeigt und gibt einen Trend der Softwarequalität an.

% Maintainability, Security, Reliability, Complexity

\subsubsection{Funktionsweise}\label{sonar:funktionsweise}
In einem produktiven Umfeld ist die Funktionsweise von SonarQube wie folgt:
SonarQube wird auf einem Server installiert und mit einem \ac{CI} Server verbunden.
Wenn Entwickler Änderungen am Code in das jeweilige \ac{SCM} System pushen löst dies einen Build auf dem \ac{CI} Server aus.
Dieser Build beinhaltet Sonar Scanner.
Ist der Build abgeschlossen werden die Berichte der Scanner an die SonarQube Instanz übermittelt.
Dort werden sie verarbeitet, in der Datenbank abgelegt und über eine Benutzerschnittstelle dargestellt \parencite{malloy_2021}.

\subsection{Qualitätssicherungstools bei der Landis+Gyr}
Für die Qualitätssicherung der Firmware setzt die Landis+Gyr aktuell C-STAT\footnote{https://www.iar.com/cstat} für statische Codeanalysen ein.
Eine kürzlich durchgeführte Evaluation hat jedoch ergeben, dass SonarQube die geeignetere Lösung wäre.
Dieser Umstieg ist zum Zeitpunkt dieser Arbeit bereits geplant, jedoch noch nicht realisiert.
Bei den verschiedenen C\# Projekten, welche in diesem Kapitel genannt wurden, werden keinerlei Qualitätsicherungstools eingesetzt.

\input{chapters/uiergonomie.tex}


