\chapter{Realisierung}
% Dies ist das Hauptkapitel Ihrer Arbeit! Hier wird die Umsetzung der eigenen Ideen und Konzepte
% (Kapitel 3) anhand der gewählten Methoden (Kapitel 4) beschrieben, inkl. der dabei aufgetretenen
% Schwierigkeiten und Einschränkungen.
Die Realisierung lässt sich in drei Phasen unterteilen.
In der ersten Phase werden grundlegende Entscheidungen getroffen und die Basis der Arbeit implementiert.
Am Ende der ersten Phase existiert ein Prototyp,
welcher bei allen Komponenten des System über die minimalen Funktionen verfügt.
Wie im Kapitel \ref{methoden} Methoden beschrieben, folgen darauf zwei weitere Phasen,
in welche neue Anforderungen des Auftraggebers umgesetzt werden.

Über das Vorgehen während dieser drei Phase wird in den folgenden Passagen beschrieben.
Einige Aspekte der Realisierung, bspw. das Datenbankschema,  wurden bewusst nicht in diese Passagen integriert,
das sie sich über die Phasen hinweg erstrecken, 
die Veränderungen über die Phasen hinweg jedoch nicht weiter relevant sind.
Über diese wird zu Schluss dieses Kapitels berichtet.

\input{chapters/phase1}
\section{Phase 2}
Der in der ersten Phase erstellte Prototyp wurde dem Auftraggeber in Form eine Demo gezeigt.
Dabei sind folgenden Rückmeldungen entstanden:
\begin{itemize}
    \item Der Prototyp entspricht den Vorstellungen und erfüllt die definierten Anforderungen.
    \item Die Vorgabe, dass die Benutzerschnittstelle für Mobilgeräte ausgelegt sein soll, fällt weg.
          Sie soll nur für Desktop-Geräte ausgelegt sein.
    \item Nebst Spannungswerten soll die Applikation auch Leistungs- und THD Werte verarbeiten können.
    \item Bisher wurden sämtliche Daten künstlich erstellt.
    Diese sollen um echte Messwerte erweiter werden, welche von Auftraggeber zur Verfügung gestellt werden.
    \item Die Benutzerschnittstelle soll um die Funktion erweitert werden,
    dass eine spezifische Zeitspanne ausgewählt werden kann.

    

\end{itemize}

\subsection{Fakemeter}

\subsection{\ac{MQTT} Datenverarbeitung}

\subsection{Benutzerschnittstelle}

\subsection{Schwierigkeiten}
Schwierigkeiten während der zweiten Phase

\section{Phase 3: Abschlussarbeiten}

\subsection{Labeling}
- Labeling:
Labels graphisch darstellen
Labels mittels UI hinzufügen

\subsection{Schwierigkeiten}
Schwierigkeiten während der dritte Phase

\subsection{Datenbankschema}

Gleich zu beginn des Projektes wurde das Datenbankschema erstellt.
Dies gibt eine gute Übersicht ob die für das Projekt benötigten Daten
abgespeicher werden können. In diesem Fall sind das die Messwerte der Stromzähler
und die Labels des Benutzers.

Die Datenbank wurde so einfach wie möglich gehalten um alle vorgegebenen
Anforderungen zu erfüllen.

\begin{figure}[h]
    \centering
    \includegraphics[width=1.0\textwidth]{gfx/smic-db}
    \caption{
        ER Diagramm der Datenbank
    }
    \label{fig:smic-db}
\end{figure}

Wie in Abbildung \ref{fig:smic-db} dargestellt, können in der Datenbank
die Verschiedenen Stromzähler mit ihrer Seriennummer abgespeichert werden.
Beispielsweise wird pro Haushalt ein Stromzähler mit einer bestimmten
Seriennummer abgespeichert, sobald dieser zum ersten mal Daten sendet.
Ein solcher Stromzähler sendet danach einzelne Messdaten zu bestimmten
Zeitpunkten. Die Messdaten beinhalten neben der Messzeit die drei Phasen
und wie viel Leistung gebraucht wurde.

Damit danach abgespeichert werden kann zu welcher Zeit beispielsweise ein
Toaster gelaufen ist, werden Labels erstellt. Ein solches Label hat eine
Bezeichnung und kann verschiedene Annotationen beinhalten.
Eine Annotation ist immer von einem bestimmten Stromzähler und kann implizit mehrere
Messdaten beinhalten. Dies wird über eine Start- und Endzeit abgespeichert.
Durch diese Struktur können danach einfach Auswertungen der Labels über mehrere
Stromzähler und Messpunkte gemacht werden.


Die zu visualisierenden Daten sind jeweils Messwerte von Spannung oder Leistung zu einem bestimmten Zeitpunkt.
Da diese Daten nur dann aussagekräftig und interessant sind, wenn sie in zeitliche Abfolge dargestellt sind.
Deshalb wurden Liniendiagramme gewählt. Ein Vortiel dieser ist, dass sicht verschiedene Werte, bspw. die drei Spannungswerte der verschiedenen Phasen
gleichzeitig dargestellt werden können. TODO Bild


