%*******************************************************
% Abstract
%*******************************************************
%\renewcommand{\abstractname}{Abstract}
\begingroup
\let\clearpage\relax
\let\cleardoublepage\relax
\let\cleardoublepage\relax

\cleardoublepage


\pdfbookmark[0]{Abstract}{Abstract}
\chapter*{Abstract}
Die vorliegende Arbeit beschriebt den Entwicklungsprozess einer Software für die Landis+Gyr.

Während der Entwicklung von intelligenten Stromzählern setzt die Landis+Gyr zu Debugging- und Testing-Zwecken eine Software ein um Werte der Zähler lesen und schrieben zu können.
Die bisherige Software ist in die Jahre gekommen und nicht benutzerfreundlich.
Sie soll durch eine Neuentwicklung ersetzt werden.
Nebst den Funktionen der bisherigen Software sollen Zusatzfunktionen basieren auf den Wünschen der Nutzer implementiert werden.

In einem ersten Schritt werden die Nutzer befragt.
Es wird erfasst, welche Aspekte der bestehenden Anwendung gut und welche schlecht sind.
Des Weiteren werden Ideen für Zusatzfunktionen gesammelt und priorisiert.
Basierend auf diesen Erkenntnissen wird iterativ eine neue Anwendung implementiert.
Eine Technologieevaluation ergab, dass diese mit C\# und WinUI3 für Windows Desktop umgesetzt wird.
Um eine hohe Benutzerfreundlichkeit zu gewährleisten, wird die Thematik «Usability» bearbeitet und die daraus gewonnenen Erkenntnisse in der Entwicklungsarbeit angewendet.
Die Qualität der Software wird mit gängigen Tools und Prozessen gewährleistet.

Für die Kommunikation mit den Zählern wird interner Code der Landis+Gyr verwendet, welcher eine tiefe Qualität aufweist.
Mit diesem Risiko muss angemessen umgegangen werden.
Obwohl die Landis+Gyr mehrere C\# Anwendungen unterhält und einsetzt, sind im Bereich CI/CD sowie Qualitätssicherung noch keine etablierten Tools aufgesetzt.
In dieser Arbeit soll dies geändert werden.

Die in der Abbildung gezeigte Anwendung «DlmsQuickAccess» baut auf den Stärken des Vorgängers auf bügelt dessen Schwächen durch neue Features aus.
Das moderne Interface der Neuentwicklung ermöglicht einen einfachen und übersichtlichen Zugriff auf die Objekte der Stromzähler. 
In den ersten Sprints wurden die Grundfunktionen implementiert.
Auf diese folgten einige Zusatzfeatures, welche in Zusammenarbeit mit den Nutzern definiert und priorisiert wurden.
Positives Feedback von den Nutzern, welche «DlmsQuickAccess» bereits in ihren täglichen Arbeiten einsetzen, bestätigt, dass Schwächen der alten Software eliminiert sowie mit neuen Funktionen ein Mehrwert geboten werden konnten.

Für diese Arbeit wurde die Zielgruppe des «DlmsQuickAccess» innerhalb der Landis+Gyr auf Firmwareentwickler am Standort Cham beschränkt.
In Zukunft soll die alte Software jedoch vollständig abgelöst und von über 100 Entwickler*innen, verteilt auf fünf Kontinente, verwendet werden.
Im Backlog befinden sich noch Work-Items, welche im Rahmen dieser Arbeit nicht bearbeitet werden konnten.
Dabei handelt es sich um neue Funktionen, welche von dem Nutzer bereits zu Beginn des Projekts gewünscht wurden, sowie Bug-Reports und Verbesserungsvorschläge, welche die Nutzer bei der Verwendung der Anwendung rückmeldeten.


TODO

\endgroup

\vfill
