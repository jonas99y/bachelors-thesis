%*******************************************************
% Abstract
%*******************************************************
%\renewcommand{\abstractname}{Abstract}
\begingroup
\let\clearpage\relax
\let\cleardoublepage\relax
\let\cleardoublepage\relax

\cleardoublepage


\pdfbookmark[0]{Abstract}{Abstract}
\chapter*{Abstract}
Die vorliegende Arbeit beschriebt den Entwicklungsprozess der Software \dq DlmsQuickAccess\dq, welche von der Landis+Gyr AG in Auftrag gegeben und von einem Bachelor Studenten der Hochschule Luzern erarbeitet wurde.

Eine veraltete Software, welche bei der Entwicklung intelligenter Stromzähler zu Debugging- und Testing-Zwecken eingesetzt wird, wird durch eine Neuentwicklung ersetzt.
Die Nutzer dieser Software werden mehrmals in den Entwicklungsprozess miteinbezogen.
Bei der Entwicklung wird speziellen Wert auf Softwarequalität und Usability gelegt.
Der aktuelle Stand der Theorie und der Technik zu diesen und weiteren Themen wird aufgearbeitet.

Des Weiteren werden Ideen, Konzepte und Methoden erklärt, welche für das Projekt relevant sind.
Eine Technologieevaluation gibt Auskunft, wieso C\# und WinUI3 für die Entwicklung eingesetzt wurden.
Das iterative Vorgehen wird detailliert beschrieben.
Nebst Entwicklungs- und Programmierarbeiten gehören dazu auch beispielsweise das Aufsetzten einer Continuous-Integration-Pipeline auf Azure DevOps oder die Qualitätssicherung mittels SonarQube.

Das erstellte Produkt wird anhand der Theorie sowie Rückmeldungen der Nutzer validiert.
Abschliessend wird festgehalten, welche Arbeiten noch offen stehen und wie die Software weiter verbessert werden kann.
\endgroup

\vfill
